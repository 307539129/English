\documentclass[10pt]{article}
\usepackage{ctex}
\usepackage{CJK}
\usepackage{graphicx}
\bibliographystyle{plain}
\setlength{\parindent}{2em}
\begin{document}
\title{Unemployment caused by automation}
\author{Qilei Zhang}
\date{may 3 2018}
\maketitle
\par
\begin{figure}[htbp]
\small
\centering
\includegraphics[width=20em]{picturea.jpg}
\caption{Figure: The development of automation is changing people's lives.}
\label{fig:lable}
\end{figure}
\par
\section{background}
Hundreds of millions of workers whose jobs are wiped out by automation between now and 2030 will still find gainful employment �� but only if governments in the countries most affected embark on massive retraining and infrastructure spending.
\par
\section{text}
About half of the work done by workers has been automated by today's technology. There may not be much work done entirely by machines, but if the remaining tasks are redistributed among fewer workers, there will still be widespread unemployment. Economic growth, the rapid application of new technologies and new jobs may have a greater positive impact.
\par
Retraining of hundreds of millions of workers in the middle of their careers will be a bigger challenge. In addition, the most affected countries, including the United States and Japan, will need to inject huge amounts of capital into infrastructure and construction to cope with the impact. If enterprises use AI and robots faster than expected, jobs will be eliminated faster.\cite{higham1994bibtex}
\bibliography{aaa}
\par
\footnote{\centering Unemployment caused by automation}
\end{document}

