\documentclass[10pt]{article}
\usepackage{ctex}
\usepackage{CJK}
\usepackage{graphicx}
\begin{document}
\title{Print still has a future}
\author{Qilei Zhang}
\date{April 27 2018}
\maketitle
\par
Pew Research Center in Washington DC sees 2015 as possibly ��the worst year for newspapers since the Great Recession��. With the development of Internet, a lot of information can be easily obtained. The paper media, which sells information and advertising as its main source of revenue, has been greatly impacted.
\par
\includegraphics[width=10em]{print.png}
\par
The successful transformation of the French Le Monde and the British economists hints that we can still get the favor of the readers with the bright and mature media. These evidences show that the media still have survival opportunities for those whose news or comments are targeted.
\par
Faced with the impact of new media and huge changes in people's reading habits, paper media needs to be reformed. The transformation of paper media must be located on the Internet media operation, in accordance with the Internet mode to allow consumers to get information quickly, and retain the content of the traditional media.
\footnote{\centering Print still has a future}
\end{document}

