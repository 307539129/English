\documentclass[twocolumn]{article}
\setlength\columnsep{30pt}
\usepackage{ctex}
\usepackage{CJK}
\usepackage{graphicx}
\bibliographystyle{plain}
\setlength{\parindent}{2em}
\begin{document}
\title{Combustible ice}
\author{Qilei Zhang}
\date{may 11 2018}
\maketitle
\par
\begin{figure}[htbp]
\small
\centering
\includegraphics[width=20em]{keran.jpg}
\caption{Figure:Combustible ice}
\label{fig:lable}
\end{figure}
\par
\section{background}
China has become the first country to exploit Natural Gas Hydrate (NGH, or Gas Hydrate) resources at sea, and the Communist Party of China (CPC) Central Committee and the State Council sent a message of congratulation to the country's team in the South China Sea, it was reported last Thursday.
\par
\section{text}
Nowadays, countries all over the world are seeking new energy sources. Due to the shortage of energy supply such as oil and natural gas, new energy sources of pollution need to be developed. The composition of flammable ice is similar to that of natural gas and there is no harmful gas emission after combustion.\cite{higham1994bibtex}
\par
Combustible ice is difficult to exploit in the deep sea. China first produced combustible ice, and it hopes to replace oil and other energy in the future. The oil pollution is large, and now the stock is also less, high flammable ice, low pollution, is an ideal clean energy in the future.
\bibliography{aaa}
\par
\footnote{\centering Combustible ice}
\end{document}

